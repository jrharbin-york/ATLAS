The API for the collective intelligence is specified as a number of functions in
Listing \ref{CI-API}. The first API function \function{startVehicle} sends the start
command to the CARS simulation. The vehicle helm is not automatically active in
a MOOS simulation, since inconsistencies in start-up time may lead to vehicles
beginning at different times. This also provides a way for the CI to start some
robots later part way through the simulation when needed. Conversely,
\function{stopVehicle} sends the stop command to the vehicle helm, and
\function{returnHome} sends the command for the vehicle to navigate to its home
location and to remain there. \function{setPatrolAroundRegion} sets the robot to
navigate in a lawn-mower style patrol (initially horizontally, then vertically)
around a particular specified rectangular region. The step size is the vertical
step used in the vertical motion steps. \var{messageName} specifies the message
name in the DSL which is used to activate the message. The function is supplied
in two versions, one of which includes a repeat count for specifying how many
times the vehicles will traverse the given waypoints before completion is
signalled. \function{setSweepAroundPoint} creates an analogous sweep pattern
around the given point. All of these waypoint navigation functions return the
set of raw coordinates that were used for the waypoint sweep, allowing them to
be stored by the CI for later use. \function{registerTimer} registers a new
timer than can be used in order to perform work later upon the CI. The timers
specified as the Timer object may be periodic or one-off, and a Java lambda
expression is used in these timers to supply the code which is to be executed
when the timer is invoked. This mechanism can be used, e.g. in the motivating
example, to return vehicles to their original sweep pattern after a certain time
for verification is expired.



\begin{figure}[t]
\begin{lstlisting}[basicstyle=\scriptsize, caption=CI API function interface, label={CI-API}]
  public static void startVehicle(String robotName);
  public static void stopVehicle(String robotName);
  public static void returnHome(String robotName);
  public static List<Point> setPatrolAroundRegion(String robotName, Region r, double stepSize, String messageName);
  public static List<Point> setPatrolAroundRegion(String robotName, Region r, double stepSize, String messageName, int repeatCount);
  public static List<Point> setSweepAroundPoint(String robotName, Point p, double size, double stepSize, String messageName);
  public static List<Point> setPatrolCoords(String robotName, List<Point> coords, String MessageName, int repeatCount);
  public static void setDepth(String robotName, double depth, String messageName);
  public static void registerTimer (String timerName, Timer timer);
\end{lstlisting}
\end{figure}



A fragment of the advanced collective intelligence for our motivating example in
Section \ref{sec:motivating-example} is presented in Listing
\ref{CI-EXAMPLE-CASESTUDY1}, which illustrates the response to a sonar sensor
detection. \function{SONARDetectionHook} is called when a sensor detection
occurs. Firstly, the code verifies if the detection is fresh, that is, if it is
an original detection has not been handled by any other vehicles. If so, the
detected classification is used to select a number of vehicles to perform
verifications according to the type of the objects. Every call to the function
\function{verifyOneRobot} selects a vehicle not currently confirming a detection
and uses an API function to reassign it to perform verifications in an area
around the detection zone. The vehicles will be reassigned to return to their
original sweep patterns using a behaviour variable, in particular the behaviour
variable \(WAYPOINT\_COMPLETE\) which will be set to true when the navigation of
these waypoints is completed.


\begin{figure}[t]
  \begin{lstlisting}[basicstyle=\scriptsize, caption=CI API fragment example for the motivating example , label={CI-EXAMPLE-CASESTUDY1}]
	public static void SONARDetectionHook(SensorDetection detection, String robotName) {
		int label = (Integer) detection.getField("objectID");
		String detectionType = (String) detection.getField("type");

		if (freshDetection(label)) {
			if (detectionType.equals("benign")) {
				for (int i = 0; i < numberOfVerificationsBenign; i++) {
					verifyOneRobot(detection, sweepRobots, robotName);
				}
			} else {
				for (int i = 0; i < numberOfVerificationsMalicious; i++) {
					verifyOneRobot(detection, sweepRobots, robotName);
				}
			}
		}
	}

  public static void verifyOneRobot(SensorDetection detection, List<String> candidateRobots,
			String detectingRobotName) {
		Point loc = (Point) detection.getField("location");
		Optional<String> rName_o = chooseRobot(loc, candidateRobots, detectingRobotName);
		if (rName_o.isPresent()) {
			String rName = rName_o.get();
			API.setSweepAroundPoint(rName, loc, SWEEP_RADIUS, VERTICAL_STEP_SIZE_CONFIRM_SWEEP,
					("UUV_COORDINATE_UPDATE_VERIFY_" + rName.toUpperCase()));
			CollectiveIntLog.logCI("Setting robot " + rName + " to verify detection");
		} else {
			CollectiveIntLog.logCI("ERROR: No robots avaiable to confirm the detection");
		}
	}
  \end{lstlisting}
\end{figure}
